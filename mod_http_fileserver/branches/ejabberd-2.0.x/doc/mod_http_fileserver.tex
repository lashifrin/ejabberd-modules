\makesubsection{modhttpfileserver}{\modhttpfileserver{}}
\ind{modules!\modhttpfileserver{}}\ind{modhttpfileserver}

This simple module serves files from the local disk over HTTP.

Options:
\begin{description}
  \titem{docroot} \ind{options!docroot}
    Directory to serve the files.
  \titem{accesslog} \ind{options!accesslog}
    File to log accesses using an Apache-like format.
    No log will be recorded if this option is not specified.
\end{description}

This example configuration will serve the files from 
the local directory \verb|/var/www|
in the address \verb|http://example.org:5280/pub/archive/|.
To use this module you must enable it:
\begin{verbatim}
{modules,
 [
  ...
  {mod_http_fileserver, [
                         {docroot, "/var/www"}, 
                         {accesslog, "/var/log/ejabberd/access.log"}
                        ]
  },
  ...
]}.
\end{verbatim}
And define it as a handler in the HTTP service:
\begin{verbatim}
{listen, 
 [
  ...
  {5280, ejabberd_http, [
                         ...
                         {request_handlers, [
                                             ...
                                             {["pub", "archive"], mod_http_fileserver},
                                             ...
                                            ]
                         },
                         ...
                        ]
  },
  ...
]}.
\end{verbatim}
